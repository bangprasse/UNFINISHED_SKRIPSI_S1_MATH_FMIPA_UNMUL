% ========================================================================
%                 ------ INPUT DATA PROPERTI DOKUMEN  ------
% ========================================================================

% Masukkan tahap naskah, pilih salah satu : {sempro/semhas/pendadaran/pengesahan}
% Pastikan ditulis dengan huruf kecil
\var{\tahap}{semhas}

% Variabel tipe naskah
% Nilai akan menyesuaikan dengan input \tahap di atas secara otomatis
\ifthenelse{\equal{\tahap}{sempro}}{
    \var{\tipe}{Proposal Skripsi}
    \Var{\Tipe}{Proposal Skripsi}
}{
    \ifthenelse{\equal{\tahap}{semhas}}{
        \var{\tipe}{Draf Skripsi}
        \Var{\Tipe}{Draf Skripsi}
    }{
        \ifthenelse{\equal{\tahap}{pendadaran}}{
            \var{\tipe}{Draf Skripsi}
            \Var{\Tipe}{Draf Skripsi}
        }{
            % Todo for "pengesahan"
            \var{\tipe}{Skripsi}
            \Var{\Tipe}{Skripsi}
        }
    }
}

% Masukkan judul dalam bahasa Indonesia di sini
\Var{\Judul}{Optimalisasi Pemilihan Urutan Syuting \textit{Scene} Film \\ Menggunakan Algoritma \textit{Particle Swarm Optimization}}
\var{\subjudul}{Studi Kasus: Film "Ketika Adzan Sudah Tidak Lagi Berkumandang"}
\var{\judul}{Optimalisasi Pemilihan Urutan Syuting \textit{Scene} Film Menggunakan Algoritma \textit{Particle Swarm Optimization} (\subjudul)}

% Masukkan data diri mahasiswa di sini
\var{\mahasiswa}{Langgeng Prassadewo Sukma Adi Winoto Basla}
\var{\nim}{1807065014}
\var{\email}{langgengprassadewo@gmail.com}
\var{\kontak}{082238691473}

% Masukkan hari dan tanggal ujian
\var{\hari}{Jumat}
\var{\tanggalujian}{07 Maret 2025}

% Masukkan tanggal penelitian
\var{\tanggalpenelitian}{07 Maret 2025}

% Variabel data pembimbing naskah ini
\var{\pembimbinga}{Wasono, S.Si., M.Si.}
\var{\nipa}{NIP. 19810712 202321 1 011}
\var{\pembimbingb}{Fidia Deny Tisna Amijaya, S.Si., M.Si.}
\var{\nipb}{NIP. 19880201 201504 1 033}

% Variabel data penguji naskah ini
\var{\pengujia}{Dr. Syaripuddin, S.Si., M.Si.}
\var{\nippa}{19740112 200012 1 002}
\var{\pengujib}{Andri Azmul Fauzi, S.Si., M.Si.}
\var{\nippb}{19920608 202321 1 023}


% Varianel data pejabat jurusan dan prodi
\var{\kajur}{Dr. Syaripuddin, M.Si.}
\var{\jabatankajur}{Ketua \jurusan, \fakultas, \institut, \kota}
\var{\korprodi}{Qonita Qurrota A\'yun, S.Si., M.Sc.}
\var{\jabatankorprodi}{Koordinator \prodi, \fakultas, \institut, \kota}


% Variabel data perguruan tinggi dan fakultas
\var{\dekan}{Dr. Dra. Hj. Ratna Kusuma, M.Si.}
\var{\nipdekan}{19630416 198903 2 002}
\var{\kota}{Samarinda}
\Var{\Kota}{Samarinda}
\var{\tahun}{2025}
\var{\tingkat}{Sarjana}
\var{\institut}{Universitas Mulawarman}
\Var{\Institut}{UNIVERSITAS MULAWARMAN}
\var{\fakultas}{Fakultas Matematika dan Ilmu Pengetahuan Alam}
\Var{\Fakultas}{FAKULTAS MATEMATIKA DAN ILMU PENGETAHUAN ALAM}
\var{\jurusan}{Jurusan Matematika}
\Var{\Jurusan}{JURUSAN MATEMATIKA}
\var{\prodi}{Program Studi S1 Matematika}
\Var{\Prodi}{PROGRAM STUDI S1 MATEMATIKA}
\var{\strata}{Sarjana Matematika}
\Var{\Strata}{SARJANA MATEMATIKA}
\var{\prasyarat}{Diajukan Sebagai Persyaratan Memperoleh Gelar \strata~pada \prodi,~\jurusan,~\institut}
