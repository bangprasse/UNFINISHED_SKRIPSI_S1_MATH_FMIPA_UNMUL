% Mengubah pengaturan nomor halaman untuk sesi BAB
\pagestyle{fancy}
\lhead{}
\chead{}
\rhead{\thepage}
\lfoot{}
\cfoot{}
\rfoot{}

% Memberikan label pada BAB ini
\renewcommand{\thechapter}{\Roman{chapter}}
\chapter{PENDAHULUAN}\label{babSatu}
\renewcommand{\thechapter}{\arabic{chapter}}
\vspace{8mm}

% Mengatur jarak spasi baris
\setstretch{1.4}

% Subbab
\section{Latar Belakang}
\vspace{-4mm}
{\frenchspacing
    % TOPIK = Penjadwalan Produksi Film
    Syuting atau kegiatan pengambilan gambar dalam proses produksi film, sering kali memiliki susunan urutan adegan-adegan (\textit{scenes}) yang tidak diambil secara terurut seperti
    urutan saat film tersebut ditayangkan \mycite{Qin:2016}.
    Salah satu faktor utama yang dipertimbangkan dalam pemilihan urutan \textit{scenes} adalah lokasi dan biaya.
    Besar biaya transportasi yang dikeluarkan berbanding lurus dengan semakin banyak \textit{scenes} dengan lokasi syuting yang berbeda.
    % Semakin banyak \textit{scene} dengan lokasi syuting yang berbeda,
    % maka biaya transportasi untuk berpindah-pindah antar lokasi syuting semakin membesar.
    Jumlah biaya transportasi yang bertambah perlu diwaspadai agar tidak menyebabkan permasalahan pada biaya operasional produksi film.

    % MASALAH = Pemilihan Urutan Scene Syuting
    Tahap produksi sebuah film merupakan proses eksekusi semua hal yang telah dipersiapkan pada tahap pra-produksi, termasuk kegiatan syuting untuk setiap \textit{scene} \mycite{Haren:2015}.
    Pelaksanaan tahap produksi mengharuskan seluruh kru dan pemain (\textit{talent}) produksi film berpindah-pindah lokasi sesuai lokasi syuting \textit{scene}.
    Akibatnya terdapat biaya tambahan berupa biaya transportasi yang dikeluarkan saat melakukan perpindahan lokasi syuting.
    Oleh karena itu, pemilihan urutan \textit{scene} syuting mempertimbangkan faktor lokasi dengan memilih total biaya pelaksanaan syuting dan biaya perpindahan antar lokasi \textit{scene} yang minimum.
    Permasalahan pemilihan urutan \textit{scene} dapat dipandang sebagai permasalahan rute terpendek dengan pengoptimalan biaya perpindahan lokasi \textit{scene} sebagai objek bahasan.
    Matematika diskret dapat diterapkan terhadap optimasi biaya dalam permasalahan rute terpendek \mycite{Qin:2016}.

    % RANCANGAN SOLUSI
    % - Matematika diskret
    Matematika diskret merupakan salah satu cabang matematika dengan ruang lingkup kajian berupa objek-objek permasalahan secara diskret \mycite{Nasir:2022}.
    Suatu objek dapat dikatakan objek secara diskret jika objek tersebut terdiri dari sejumlah berhingga anggota dan tiap anggotanya berbeda atau tidak terhubung.
    Salah satu penerapan dari matematika diskret adalah sistem penyimpanan informasi pada komputer digital yang disimpan dalam bentuk diskret.
    Cakupan dalam matematika diskret mempelajari diantaranya himpunan, relasi dan fungsi, induksi matematika, kombinatorial, dan teori graf \mycite{Yurinanda:2023}.

    % - Teori Graf
    Teori graf termasuk ke dalam cakupan matematika diskret. Konsep teori graf pertama kali diperkenalkan oleh Leonhard Euler saat memodelkan solusi
    untuk melewati ketujuh jembatan K$\ddot{o}$nigsberg di Rusia dengan masing-masing jembatan hanya boleh dilewati tepat satu kali dan kembali ke tempat semula perjalanan dimulai.
    Sebuah graf $G$ merupakan pasangan terurut himpunan berhingga yang tidak kosong dari simpul (\textit{vertices}) dan himpunan pasangan dari simpul-simpul secara tidak terurut.
    Implementasi teori graf sering digunakan untuk memecahkan permasalahan optimasi, seperti optimasi pencarian rute terpendek pada konsep \textit{traveling salesman problem} (TSP) \mycite{Setiawati:2023}.

    % = Optimasi
    Optimasi atau optimalisasi merupakan suatu prinsip pencarian solusi layak yang bersifat tepat, efektif, dan efisien (optimum) \mycite{Atiqoh:2020}.
    Pengertian ptimasi juga dapat diartikan sebagai tindakan tersistematis yang memaksimalkan sumber daya yang tersedia dengan tujuan menemukan solusi optimal.
    Penerapan optimasi sering diterapkan dalam banyak permasalahan yang memiliki keterbatasan sumber daya, seperti permasalahan pencarian rute terpendek pada konsep \textit{traveling salesman problem} (TSP).

    % - Travelinging Salesman Problem (TSP)
    \textit{Traveling Salesman Problem} (TSP) merupakan konsep rute terpendek pada seorang penjual (\textit{salesman}) yang harus mengunjungi semua kota dan harus kembali ke kota awal serta satu kota hanya boleh dikunjungi satu kali \mycite{Rahimi:2023}.
    Permasalahan optimasi TSP termasuk ke dalam persoalan optimasi kombinatorial dengan memanfaatkan graf sebagai representasi dari model permasalahan.
    Representasi masalah TSP umumnya dimodelkan ke dalam graf dengan simpul (\textit{vertices}) sebagai lokasi dan sisi (\textit{edge}) sebagai bobot atau biaya yang akan dioptimalkan.
    Permasalan TSP dapat diselesaikan menggunakan penerapan algoritma \textit{Particle Swarm Optimization} (PSO).

    % - Algoritma Particle Swarm Optimization
    \textit{Particle Swarm Optimization} (PSO) merupakan algoritma optimasi stokasi terinspirasi dari perilaku sosial burung atau ikan yang berkelompok dalam mencari makan \mycite{Azhari:2018}.
    Algoritma \textit{Particle Swarm Optimization} (PSO) ini termasuk ke dalam algoritma heuristik berbasis populasi.
    Pencarian solusi optimal menggunakan perilaku partikel dalam populasi itu sendiri.
    Setiap partikel berperilaku sebagai pencari lingkungan dengan kecerdasan untuk saling berkontribusi dan berkomunikasi dalam mencari letak objek bahasan.

    % GAP PENELITIAN
    Penelitian sebelumnya tentang algoritma \textit{Particle Swarm Optimization} (PSO) telah dilakukan oleh beberapa peneliti terdahulu, antara lain \textcite{Azhari:2018} dalam penelitian yang membahas tentang penentuan rute penjemputan angkutan sekolah di MI Salafiyah Kasim.
    Dengan 2 kloter pengantaran, 150 partikel, 35 iterasi, dan 5 kali percobaan yang dihitung menghasilkan 3 percobaan dengan rute yang lebih optimal dari rute aktual.
    \textcite{Natalia:2019} dalam penelitian yang membahas tentang penentuan jarak penjemputan penumpang optimal di CV. Eira Saudara.
    Dengan 8 titik lokasi penjemputan dan 90 partikel yang dihitung menghasilkan rute optimal dengan total jarak 59,2 Km.
    \textcite{Dalyono:2017} dalam penelitian yang membahas tentang penentuan rute kunjungan tempat pariwisata di Kota Bandung.
    Dengan 2 set lokasi pariwasata dihitung dengan 2 kali pengujian, yaitu pengujian pertama dilakukan dengan mengubah 5 parameter pada set pertama dan pengujian kedua membandingkan waktu tempuh luaran antara algoritma \textit{Particle Swarm Optimization} (PSO) dan algoritma \textit{Artificial Immune System} (AIS).

    % KEBARUAN 
    Pada penelitian ini, permasalahan pemilihan urutan \textit{scene} syuting dapat dipandang ke dalam permasalahan optimasi dengan konsep TSP.
    Sehingga penulis akan menggunakan algoritma \textit{Particle Swarm Optimization} (PSO) untuk mencari solusi dalam pemilihan urutan \textit{scene} syuting yang optimal pada film "Ketika Adzan Sudah Tidak Lagi Berkumandang".
    Film "Ketika Adzan Sudah Tidak Lagi Berkumandang" merupakan film berkategori film layar lebar dengan genre horor.
    Film ini diproduksi oleh komunitas East Borneo Film (EBF).
    Proses produksi film ini menggunakan tiga lokasi utama yang berbeda, yaitu Waduk Tenggarong, Desa Loa Raya, dan Desa Kedang Ipil.
    Setiap lokasi utama terdapat beberapa titik lokasi yang menjadi lokasi syuting \textit{scene} yang berbeda.
    Hal ini memerlukan strategi dalam pemilihan urutan \textit{scene} syuting guna menjaga peningkatan biaya transportasi agar biaya produksi tetap minimum.

    % KESIMPULAN SOLUSI 
    Berdasarkan uraian di atas, penulis akan membahas mengenai masalah pemilihan urutan \textit{scene} syuting dengan menggunakan algoritma \textit{Particle Swarm Optimization} (PSO) dengan judul
    "\judul".
}
\vspace{-4mm}

% Subbab
\section{Batasan Masalah}
\vspace{-4mm}
{\frenchspacing
    Batasan masalah yang digunakan pada penelitian ini untuk mencapai tujuan yang diharapkan adalah:

    \begin{enumerate}[left = 0em, align = left, nolistsep]
        \item Data yang digunakan adalah data dari \textit{Master Breakdown} film "Ketika Adzan Sudah Tidak Lagi Berkumandang".
        \item Data lokasi yang digunakan adalah data lokasi syuting dari setiap \textit{scene}.
        \item Asumsi yang digunakan antara lain:
              \begin{enumerate}[align = left, label=(\alph*), nolistsep]
                  \item biaya perpindahan antar setiap \textit{scene} merupakan total biaya transportasi ditambahkan biaya talent yang bermain pada \textit{scene} tujuan,
                  \item biaya transportasi merupakan jumlah konsumsi bahan bakar dalam Rupiah,
                  \item kendaraan berjumlah tiga, yaitu satu kendaraan untuk seluruh pemain (\textit{talent}) dan dua kendaraan untuk perlengkapan produksi,
                  \item harga bahan bakar yang digunakan adalah harga bahan bakar jenis Pertalite per bulan Juni 2024, yaitu seharga Rp. $10.000,-$.
              \end{enumerate}
        \item Graf yang digunakan dalam penelitian ini adalah graf tak berarah.
    \end{enumerate}
}
\vspace{-4mm}


% Subbab
\section{Rumusan Masalah}
\vspace{-4mm}
{\frenchspacing
    Berdasarkan latar belakang yang telah diuraikan, rumusan masalah pada penelitian ini adalah:

    \begin{enumerate}[left = 0em, align = left, nolistsep]
        \item Bagaimana model graf \textit{scene} film "Ketika Adzan Sudah Tidak Lagi Berkumandang"?
        \item Bagaimana penerapan algoritma \textit{Particle Swarm Optimization} (PSO) untuk optimalisasi pemilihan urutan \textit{scene} syuting film "Ketika Adzan Sudah Tidak Lagi Berkumandang"?
        \item Bagaimana hasil optimalisasi pemilihan urutan \textit{scene} syuting film "Ketika Adzan Sudah Tidak Lagi Berkumandang" menggunakan algoritma \textit{particle swarm optimization} (PSO)?
    \end{enumerate}
}
\vspace{-4mm}

% Subbab
\section{Tujuan Penelitian}
\vspace{-4mm}
{\frenchspacing
    Berdasarkan rumusan masalah yang telah diuraikan, maka tujuan yang diharapkan dapat dicapai pada penelitian ini adalah:

    \begin{enumerate}[left = 0em, align = left, nolistsep]
        \item Mengetahui model graf \textit{scene} film "Ketika Adzan Sudah Tidak Lagi Berkumandang".
        \item Mengetahui penerapan algoritma \textit{Particle Swarm Optimization} (PSO) untuk optimalisasi pemilihan urutan \textit{scene} syuting film "Ketika Adzan Sudah Tidak Lagi Berkumandang".
        \item Mengetahui hasil optimalisasi pemilihan urutan \textit{scene} syuting film "Ketika Adzan Sudah Tidak Lagi Berkumandang" menggunakan algoritma \textit{particle swarm optimization} (PSO).
    \end{enumerate}
}
\vspace{-3mm}

% Subbab
\section{Manfaat Penelitian}
\vspace{-4mm}
{\frenchspacing
    Manfaat yang diharapkan pada penelitian ini adalah:

    \begin{enumerate}[left = 0em, align = left, nolistsep]
        \item Bagi penulis dan pembaca, dapat memberikan pemahaman lebih luas mengenai penerapan algoritma \textit{Particle Swarm Optimization} (PSO).
        \item Bagi universitas, dapat memberikan kontribusi pengembangan ilmu pengetahuan serta dapat menjadi bahan referensi bagi mahasiswa lain.
        \item Bagi pihak terkait, dapat memberikan pemahaman mengenai penerapan algoritma \textit{Particle Swarm Optimization} (PSO) dan dapat digunakan untuk membantu pemilihan urutan \textit{scene} syuting film yang akan mendatang.
    \end{enumerate}
}
\vspace{-3mm}

% Memotong halaman
\pagebreak