% Memberikan label pada BAB ini
\renewcommand{\thechapter}{\Roman{chapter}}
\chapter{METODE PENELITIAN}\label{babTiga}
\renewcommand{\thechapter}{\arabic{chapter}}
\vspace{8mm}

% SubBab
\section{Waktu dan Tempat Penelitian}
\vspace{-4mm}
{\frenchspacing
    Penelitian ini dilaksanakan pada bulan Maret 2025 sampai Mei 2025.
    Pengambilan data dilakukan pada film "Ketika Adzan Sudah Tidak Lagi Berkumandang".
    Pengolahan data dalam penelitian ini dilakukan di Laboratorium Matematika Dasar dan Laboratorium Matematika Komputasi yang terletak di
    Fakultas Matematika dan Ilmu Pengetahuan Alam, Universitas Mulawarman, Samarinda, Kalimantan Timur.
}
\vspace{-10mm}

% SubBab
\section{Variabel Penelitian}
\vspace{-4mm}
{\frenchspacing
    Variabel yang digunakan dalam penelitian ini adalah lokasi syuting setiap \textit{scene} dan biaya perpindahan antar setiap \textit{scene} pada film
    "Ketika Adzan Sudah Tidak Lagi Berkumandang".
    Penelitian ini memfokuskan pada pemilihan urutan syuting \textit{scene} dengan konsep \textit{traveling salesman problem} menggunakan algoritma \textit{particle swarm optimization} sehingga tahap syuting setiap \textit{scene} berjalan lebih efektif dan efisien.
    Variabel yang digunakan dapat dilihat pada tabel berikut:

    \begin{table}
        \centering
        \caption{Variabel Penelitian}
        \label{tab: variabel penelitian}
        \begin{tabular}{|l|l|l|}
            \hline
            \multicolumn{1}{|c|}{\textbf{Variabel}} & \multicolumn{1}{c|}{\textbf{Keterangan}}   & \multicolumn{1}{c|}{\textbf{Satuan}} \\ \hline
            $v_{i}$                                 & Titik atau lokasi ke-$i$                   & -                                    \\ \hline
            $v_{i}v_{j}$                            & Biaya antara titik ke-$i$ dan titik ke-$j$ & Rupiah (Rp)                          \\ \hline
        \end{tabular}
    \end{table}
}
\vspace{-3mm}

% SubBab
\section{Teknik Pengumpulan Data}
\vspace{-4mm}
{\frenchspacing
    Teknik pengumpulan data yang digunakan dalam penelitian ini adalah menggunakan data sekunder.
    Data sekunder yang diambil dari \textit{Master Breakdown} film "Ketika Adzan Sudah Tidak Lagi Berkumandang" terdiri dari lokasi syuting setiap \textit{scene}, pemain (\textit{talent}) di setiap \textit{scene},
    dan biaya setiap pemain (\textit{talent}).
    Sedangkan data sekunder yang diambil dari aplikasi Google Maps adalah jarak dari lokasi syuting setiap \textit{scene}.
}
\vspace{-3mm}

% SubBab
\section{Populasi dan Sampel Penelitian}
\vspace{-4mm}
{\frenchspacing
    Populasi pada penelitian ini adalah rute dan biaya perpindahan antar setiap \textit{scene} pada film "Ketika Adzan Sudah Tidak Lagi Berkumandang".
    Sampel yang digunakan pada penelitian ini sama dengan populasi penelitian, yaitu seluruh rute dan biaya perpindahan antar setiap \textit{scene} pada film "Ketika Adzan Sudah Tidak Lagi Berkumandang".
}
\vspace{-3mm}

% SubBab
\section{Teknik Sampling}
\vspace{-4mm}
{\frenchspacing
    Teknik sampling yang digunakan dalam penelitian ini adalah teknik \textit{purposive sampling}.
    Teknik \textit{purposive sampling} merupakan teknik penentuan sampel dengan mempertimbangankan maksud dan tujuan tertentu karena sampel tersebut memiliki informasi yang diperlukan \mycite{Amin:2023}.
    Pada penelitian ini sampel yang digunakan sama dengan populasi karena bertujuan untuk mengambil seluruh data pada film "Ketika Adzan Sudah Tidak Lagi Berkumandang".
}
\vspace{-3mm}

% SubBab
\section{Teknik Analisis Data}
\vspace{-4mm}
{\frenchspacing
    Teknik analisis data pada penelitian ini menggunakan konsep \textit{traveling salesman problem} dengan metode algoritma \textit{particle swarm optimization}.
    Adapun tahapan yang digunakan sebagai berikut:

    \begin{enumerate}[align=left, left=0mm, nolistsep]
        \item Mengumpulkan dan mempersiapkan data, dengan tahapan sebagai berikut: \par \nobreak
              \begin{enumerate}[label=(\alph*), align=left, left=0mm, nolistsep]
                  \item Mengumpulkan data lokasi syuting setiap \textit{scene}, pemain (\textit{talent}) di setiap \textit{scene}, biaya pemain (\textit{talent}) di setiap \textit{scene}, dan jarak antar lokasi syuting setiap \textit{scene}.
                  \item Menghitung biaya perpindahan antar setiap \textit{scene}.
              \end{enumerate}
        \item Studi Literatur \par \nobreak
              Studi literatur dilakukan untuk mendapatkan informasi dari buku-buku dan catatan penelitian terdahulu mengenai konsep \textit{traveling salesman problem} dengan metode algoritma \textit{particle swarm optimization}.
              Informasi tersebut akan menjadi pendukung dari ide dalam penyelesaian masalah.
        \item Merumuskan masalah dan variabel penelitian
        \item Melakukan tahap inisialisasi, dengan tahapan sebagai berikut: \par \nobreak
              \begin{enumerate}[label=(\alph*), align=left, left=0mm, nolistsep]
                  \item Menentukan jumlah iterasi maksimum $(n)$.
                  \item Menentukan ukuran \textit{swarm} $(N)$ dan dimensi partikel $(d)$.
                  \item Menentukan \textit{learning rates} ($c_{1}$ dan $c_{2}$).
                  \item Menentukan batas atas $(X_{max})$ dan batas bawah $(X_{min})$ dari posisi partikel.
                  \item Menentukan \textit{velocity} maksimum $(V_{max})$.
                  \item Membangkitkan \textit{velocity} awal $(\textbf{\textit{V}}_{j}^{(0)})$ sejumlah $N$ partikel.
                  \item Membangkitkan posisi awal $(\textbf{\textit{X}}_{j}^{(0)})$ sejumlah $N$ partikel.
                  \item Mengevaluasi hasil rute setiap partikel.
                  \item Mengevaluasi nilai fungsi \textit{fitness} setiap partikel.
                  \item Menentukan $P_{best,j}^{(0)}$ setiap partikel dan $G_{best}^{(0)}$.
              \end{enumerate}
        \item Melakukan tahap perulangan, dengan tahapan sebagai berikut: \par \nobreak
              \begin{enumerate}[label=(\alph*), align=left, left=0mm, nolistsep]
                  \item Membangkitkan \textit{inertia weight} $(\omega^{(i)})$.
                  \item Membangkitkan $r_{1}^{(i)}$ dan $r_{2}^{(i)}$.
                  \item Memperbarui \textit{velocity} setiap partikel.
                  \item Memperbarui posisi setiap partikel.
                  \item Mengevaluasi hasil rute setiap partikel.
                  \item Mengevaluasi nilai fungsi \textit{fitness} setiap partikel.
                  \item Menentukan $P_{best}^{(i)}$ setiap partikel dan $G_{best}^{(i)}$.
                  \item Mengevaluasi kriteria pemberhentian.
              \end{enumerate}
        \item Menarik kesimpulan urutan syuting \textit{scene} menggunakan hasil rute terbaik dari $G_{best}$.
    \end{enumerate}

}
\vspace{-3mm}

\pagebreak
% SubBab
\section{Kerangka Penelitian}
\vspace{-4mm}
{\frenchspacing
    Kerangka dari penelitian ini disajikan dalam \textit{flowchart} sebagai berikut:

    \begin{figure}
        \centering
        \begin{tikzpicture}[node distance = 2.5cm]
            \node (start) [startstop] {Mulai};
            \node (input1) [inout, below of = start] {Data};
            \node (proc1) [process, below of = input1] {Studi Literatur};
            \node (proc2) [process, below of = proc1] {Merumuskan Masalah};
            \node (proc3) [process, below of = proc2] {Membuat Model Matematika \textit{Traveling Salesman Problem}};
            \node (proc4) [process, below of = proc3] {Menyelesaikan Model Matematika \textit{Traveling Salesman Problem} \\ dengan Algoritma \textit{Particle Swarm Optimization}};
            \node (output1) [inout, below of = proc4] {Memperoleh Urutan Syuting \textit{Scene} Optimal \\ Berdasarkan Hasil Rute Terbaik};
            \node (stop) [startstop, below of = output1] {Selesai};

            \draw [arrow] (start) -- (input1);
            \draw [arrow] (input1) -- (proc1);
            \draw [arrow] (proc1) -- (proc2);
            \draw [arrow] (proc2) -- (proc3);
            \draw [arrow] (proc3) -- (proc4);
            \draw [arrow] (proc4) -- (output1);
            \draw [arrow] (output1) -- (stop);

        \end{tikzpicture}
        \vspace{5mm}
        \caption{\textit{Flowchart} Kerangka Penelitian}
        \label{gam: Kerangka Penelitian}
    \end{figure}
}
\vspace{-3mm}